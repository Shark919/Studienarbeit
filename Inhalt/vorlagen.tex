\chapter{Grundlagen}
\label{cha:Fundamentals}

\section{AR.Drone 2.0}
%Christoph
Bei der AR.Drone 2.0 handelt es sich um einen ferngesteuerten Quadrocopter des französischem Herstellers Parrot SA. \cite{drone} Die Drohne ist standardmäßig steuerbar mit einer mobilen Applikation für Android und iOS Geräte. Dafür baut sie ein WLAN Netzwerk auf, mit dem sich die Geräte verbinden können. Zur Steuerung stellt die AR.Drone ein Interface zur Verfügung, mit dem sie ferngesteuert werden kann. \newline
Im Umfang der Studienarbeit wird die aktuellste Version der AR.Drone 2.0 verwendet. Diese zeichnet sich unter Anderem durch eine Frontkamera mit einer Auflösung von 1280×720 Pixeln und einer Bildrate von 30 fps aus. Weiterhin ist Sie mit einer zum Boden gerichteten QVGA Kamera ausgerüstet, welche 60 Bilder pro Sekunde aufnimmt. \newline
Die Drohne orientiert sich beim Fliegen mit Hilfe einer Vielzahl von Sensoren. Dazu gehören ein dreiachsiges Gyroskop und ein Magnetometer. Weiterhin nutzt sie Beschleunigungs-, Ultraschall- und Luftdrucksensoren. \newline
Der Grund für die Wahl der Drohne ist vor allem der vergleichsweise niedrige Preis von ca. 200€ und der starken Verbreitung in der Forschung. Dadurch gibt es bereits eine Vielzahl von Projekten, die dazu führen, dass die Drohne und das dazugehörige Interface zu einem großen Umfang fehlerfrei funktionieren. \newline
Weiterhin gibt es schon ROS Nodes (siehe \ref{ROS}) und konfigurierte Modelle in Simulationsumgebungen, welche die Arbeit an dem Projekt beschleunigen.


\section{ROS}
%Max
\label{ROS}

\section{Simulation}
%Max

\section{Fuzzylogik}
%Würde erwähnen dass wir es verwenden
%wollen wir das überhaupt reinnehmen? haben wir ja eigentlich gar nicht gemacht und mussten wir uns auch %null mit beschäftigen


\section{Kinect}
%Max