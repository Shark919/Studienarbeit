\fancypagestyle{plain}{%
\fancyhf{} % clear all header and footer fields
\fancyhead[R]{\nouppercase{\leftmark}}
\fancyfoot[C]{\thepage} % except the center
%\fancyfoot[R]{Christoph Meise}
\renewcommand{\headrulewidth}{0.4pt}
\renewcommand{\footrulewidth}{0pt}} 


\pagestyle{plain}
\chapter{Einleitung}
\label{cha:Introduction}
%Christoph
Autonomes Fahren, Machine Learning und Industrie 4.0. Hinter diesen aktuellen Themen steckt das Ziel, Abläufe und Zusammenhänge kontinuierlich zu automatisieren und für den Menschen zu vereinfachen. Das Thema der Automation kann vor allem in der Interaktion zwischen Mensch und Maschine hilfreich sein. Auf Grund der geringen Kosten und der hohen Anwendungsvielfalt bieten sich vor allem Drohnen für die Forschung zu Automation in der Robotik an. \newline
Mit Modellen ab 30 und bis zu mehreren tausend Euro gibt es Ausführungen für nahezu jeden Anwendungsfall. Meist mit mehreren Sensoren und Kameras ausgestattet, stellen sie nicht nur Spielzeug dar, sondern sind essentiell für reale Anwendungsgebiete. \newline
So werden heute schon Drohnen genutzt, um Katastrophengebiete und Kriegsregionen aus sicheren Standorten aufzuklären, oder um die Feuerwehr bei der Branderkundung und Menschensuche zu unterstützen. \newline
Natürlich können sie auch genutzt werden, um alltägliche Probleme zu lösen, wie die schnelle und direkte Lieferung von Paketen. \newline
Da diese große Zahl an Drohnen nicht mehr manuell gesteuert werden kann, müssen sich diese größtenteils autonom bewegen. Dabei treten eine Vielzahl von komplexen Problemen auf, wie das zurechtfinden in einem unbekannten Raum und die Objekterkennung. \newline
Außerdem ist bei Fluggeräten das Problem, dass die verwendete Ausrüstung tendenziell leicht und klein sein muss, damit die Flugeigenschaften nicht eingeschränkt werden bzw. die Drohne nicht zu groß wird. \newline
Im Umfang dieser Arbeit soll eine Vorstufe zum autonomen Fliegen untersucht und implementiert werden: ein Assistenzsystem für den manuellen Flug. Dies ist Vergleichbar mit den Assistenzsystemen in PKWs, bei denen ein Tempomat, Licht- und Regensensoren oder Spurhalteassistenten den Fahrer unterstützen, jedoch das Fahren nicht abnehmen. \newline
Eine solche semi-autonome Assistenzfunktion könnte beispielsweise dabei helfen, die Drohne durch enge Räume und durch Hindernisse zu fliegen, also Flugmanöver, bei denen ein Mensch eine hohe Fehleranfälligkeit aufweist. \newline


\section{Motivation}
%Christoph
Das Ziel besteht darin, dass die Drohne aktiv die Umgebung auswertet und dabei Objekte wie Türen, oder Wände erkennt. Der Unterschied zu bereits bestehenden Projekten in diesem Themengebiet besteht darin, dass nur eine einzelne monokulare Kamera verwendet werden soll, anstatt externe Tiefenbildkameras zu montieren. \newline
Dadurch trifft man auf eine Vielzahl von komplexen Problemen, welche im weiteren Verlauf dieser Arbeit dargestellt werden. Auf der anderen Seite könnte man dadurch teure Hardware sparen und somit auch auf andere Projekte anwenden. \newline
Anhand der Tiefenbilder soll die Drohne anschließend in der Lage sein, die Entscheidungen des Nutzers zu unterstützen und abzuändern, um somit beispielsweise Kollisionen zu vermeiden, oder gezielt durch Hindernisse zu fliegen. \newline
Das Problem soll nur durch die integrierte Hardware gelöst werden, also einer einfachen 720p Kamera die nach vorn gerichtet an der Drohne befestigt ist. Weiterhin ist das Ziel der Arbeit eine vollständige Modularisierung der Softwarearchitektur. Dies ist essentiell, da das Assistenzsystem sowohl bei einer realen Drohne, als auch in einer Simulation funktionieren soll. 
%todo: gestensteuerung mit erwähnen
\section{Aufbau}
%Christoph
Diese Studienarbeit besteht aus 3 Kapiteln. Im ersten Abschnitt der Arbeit sollen die Grundlagen erklärt werden. Zuerst wird dabei die genutzte Drohne und deren Spezifikationen dargestellt. \newline
Anschließend wird das Software Framework Roboter Operating System \emph{(ROS)} eingeführt, welches ein Hauptbestandteil der Projektarchitektur ausmacht.\newline
Im weiteren Verlauf wird dann die Simulationsumgebung beschrieben, in der eine simulierte Drohne geflogen werden kann.
Im letzten Teil dieses Kapitels wird die Tiefenbildkamera Kinect beschrieben, welche zur Umsetzung der Gestensteuerung genutzt wird. \newline
Der zweite Abschnitt dieser Arbeit umfasst die Software Architektur. Zuerst werden hierbei die technischen und Anforderungen an das Projekt aufgestellt. Anschließend ist das Kapitel in drei logische abgegrenzte Unterkapitel unterteilt: Bildverarbeitung, Implementierung und das Assistenzsystem. \newline
Im Teil Bildverarbeitung wird beschrieben, wie aus den einfachen Bildern zuerst die Position der Kamera im Raum approximiert wird, um anschließend aus aufeinanderfolgenden Aufnahmen Tiefeninformationen zu gewinnen. \newline
Im Folgenden wird im Abschnitt Implementierung die softwaretechnische Umsetzung und die Architektur beschrieben. Abschließend dient das Unterkapitel Assistenzsystem dazu, eine allgemeine Hinführung zu bieten und mögliche Implementierungstechniken vorzuschlagen. \newline
Der Letzte Teil der Arbeit ist die Evaluation. Diese besteht aus dem erzielten Ergebnis, sowie aus dem Ausblick für weitere Betrachtungen dieser Problematik. Dabei werden außerdem aufgetretene Probleme und Hindernisse detailliert beschrieben.

