\fancypagestyle{plain}{%
\fancyhf{} % clear all header and footer fields
\fancyhead[R]{\nouppercase{\leftmark}}
\fancyfoot[C]{\thepage} % except the center
%\fancyfoot[R]{Christoph Meise}
\renewcommand{\headrulewidth}{0.4pt}
\renewcommand{\footrulewidth}{0pt}} 


\pagestyle{plain}
\chapter{Introduction}
\label{cha:Introduction}

SAP is a multinational corporation that develops enterprise software to manage
business operations and customer relations. \cite{company} The company is
headquartered in Walldorf, Germany with the nearby location St. Leon Rot. As a
result the site is distributed among 27 buildings which are partially not within
walking distance.
Therefore a shuttle service is offered in order to simplify traveling between
the offices.
\newline The service currently works similar to public transportation:
the shuttles have a strict schedule as well as designated stops.
Furthermore it is possible to request a shuttle at the reception in some
buildings.
The drivers are therefore called by telephone and may dynamically adjust their
routes if they are available. \newline Even though this is a plausible concept,
it is to be taken into consideration that it is currently barely ever used as it
is too inconvenient for most people.
Despite this, the otherwise strict schedule causes a number of significant
problems such as long waiting times. Thereby it is highly ineffective that the
drivers are heading towards stops where possibly no one is waiting.
Moreover, if many people are simultaneously looking for a shuttle and additional
ones are necessary, the drivers have to autonomously communicate with each other
to make enough cars available. As a consequence, this process can be very
time-consuming for the passengers.
\newline 

\section{Motivation}
The objective of this study is to investigate the possible effects of an
extension of the shuttle services with smart devices and IoT solutions. This
process involves the use of the Internet of Things concept by connecting the
cars, drivers and passengers with smart devices such as sensors and smartphone
applications.
\newline One of those devices is a beacon installed in the interior of the
shuttles as well as in waiting areas in each building. A Beacon is a small
wireless device continuously transmitting a basic radio signal. Most commonly
this signal is picked up by nearby smartphones using Bluetooth Low Energy
technology. \cite{beacon1}\cite{beacon2} Moreover, technical data like current
speed, acceleration or fuel consumption are constantly being sent to the SAP
Vehicle Insights backend in real time which is explained in section \ref{SAPVI}.
The transmission is realized with a small device that is attached to the
Controller Area Network \emph{(CAN)} bus interface of the cars. This dongle
sends the data over a mobile Internet connection.
\newline The shuttles shall be solicited with the passenger application sending
a request to the backend. There it is processed and forwarded to the driver
showing him where he may pick them up.
In order to have an overview and to administrate the workflow, a real-time
dashboard is to be developed with current User Interface \emph{(UI)}
technologies.
The data that is gathered by the different devices is persitent in a database
and may be analyzed in the backend. As a result, the extrapolated information
could help to optimized the shuttle service, possibly leading towards lower
waiting times, reduced distances driven and a lower fuel consumption overall.
\newline

\section{Structure}
This paper is composed of three chapters. The first section describes
the fundamental principles of the Internet of Things. Moreover it examines SAPs
IoT strategy and its components that are important for the further course of
this thesis. \newline The following section is introducing technical as well as
non-technical requirements. Moreover it involves the design and implementation
of the architecture for the Proof of Concept \emph{(PoC)}. In this context, the
decisions for and against the utilization of software structures and elements
are justified.
\newline The next chapter is comprised of the realization of the Java interface
which evaluates the stability and therefore the applicability of this component.
Based on that, the implementation details are elucidated extensively. Notably,
this section also explains how the component has been tested, validating that it
is reasonable to use it as a core element in the architecture.
\newline The last part outlines this paper in order to analyze and evaluate the
results. As a conclusion this chapter contains a prospect on how shuttles
services may be further optimized in the future. It also includes a reflection
about the advantages the digital transformation offers in comparison to the
existing model in order to measure to which extend the architecture is
practical.
