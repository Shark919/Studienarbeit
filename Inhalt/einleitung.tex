\fancypagestyle{plain}{%
\fancyhf{} % clear all header and footer fields
\fancyhead[R]{\nouppercase{\leftmark}}
\fancyfoot[C]{\thepage} % except the center
%\fancyfoot[R]{Christoph Meise}
\renewcommand{\headrulewidth}{0.4pt}
\renewcommand{\footrulewidth}{0pt}} 


\pagestyle{plain}
\chapter{Einleitung}
\label{cha:Introduction}

- Zeitalter der Drohnen bla bla

- Drohnen schon ab 30€ bis zu mehreren tausend Euro
- nicht nur Spielzeug, reale Anwendungsgebiete
-> Erkundung in Katastrophen / Kriegsregionen
-> Feuerwehr Branderkundung und Menschensuche
-> Lieferung von Paketen etc. etc.

- Schritt von fliegen zu fliegen lassen
- in Robotik Kernproblem sich in Umgebung autonom zurechtzufinden
- Problematik von Gewicht -> keine schweren high end Geräte möglich
- meist Tiefenbildkameras; Bilder werden dann ausgewertet
- Möglichkeit der Schwarmintelligenz

\section{Motivation}

- vorhandene Drohne semiautonom fliegen lassen
-> Assistenzsystem für Nutzer => Drohne erkennt Wände / Türen und hilft beim lenken
- Problematik der Hardware => nur Kamera nach unten und vorne vorhanden
- da jeweils nur eine "Normale" Kamera kein Tiefenbild vorhanden
-> nutzen des Projekts "REMODE" von Davide Scaramuzza um aus einzelbildern Tiefenbilder zu generieren


\section{Aufbau}
