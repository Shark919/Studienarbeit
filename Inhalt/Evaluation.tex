\chapter{Evaluation}
\section{User Experience}


\section{Conclusion}
All things considered, the project proofed that the use of IoT technology was
able to enhance the shuttle service. The room for optimization is significant
especially with high amounts of data gathered by the smart devices.
\newline The proof of concept was successfully designed and implemented
manifesting the practicability of the underlying idea. With immense amount of
data about the shuttles, almost any aspect may be innovated. \newline Currently
this is limited by the absence of sensors and smart devices to gather more
detailed information such as weather data. Prospectively a connection to the
outlook calendars of SAP employees could help to predict when shuttle requests
are most likely. \newline However, the fields of application for IoT are principally
everywhere. The digital transformation allows new business models to come up and
can raise existing ones to the next level. Similarly, there is much room for
improvement for the shuttle service sector. The introduced architecture creates
a good basis and can be further extended with more smart devices and better
analytical operations in the backend, in order to innovate the service more than
ever. 
