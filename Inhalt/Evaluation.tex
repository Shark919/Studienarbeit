\chapter{Evaluation}
\section{Ergebnis}
%beide

%hier sollten wir vor allem die Probleme von SVO und REMODE ansprechen

\section{Ausblick}
\subsection{Andere Simulatoren}
Die Simulationsumgebung Gazebo ist nicht die einzige verfügbare zur Simulation von Quadrocoptern. Jedoch bringt sie durch die einfache Anbindung von ROS einiges an Vorteilen mit sich. Damit gehen allerdings auch Nachteile einher. So sind die Kameraeingaben nicht sehr realistisch und Szenarien die in der Simulation funktionieren müssen in der Realität nicht funktionieren. Ebenso ist das Flugverhalten in manchen Situation nicht realitätsgetreu und kann zu verfälschten Ergebnisse führen. Um diesem Vorzubeugen ist es ratsam die Resultate mit anderen Simulationen vergleichen. Eine aktuelle Simulationsumgebung zur Simulation von Quadrocoptern ist der AirSim von Microsoft.\cite{airsim} Ursprünglich entwickelt um Trainingsdaten zum maschinellem Lernen sammeln, kann er ebenfalls auch für herkömmliche Simulation verwendet werden. Aktuell ist allerdings nur für Windows Betriebssysteme verfügbar. \cite{airsimpaper} Er biete eine fotorealistische Umgebung und ein akkurates Flugverhalten, dadurch ist besonders für Demos und Showcases besser geeignet. \newline
Es existieren weiterhin andere Simulatoren, allerdings sind die meisten spielerisch veranlagt und bieten keine programmatische Schnittstelle, weshalb sie für den Zweck der Studienarbeit nicht sinnvoll verwendet werden könne
%beide
%besserer simulator
%genaue Untersuchung mit PCL algorithmen
%svo und remode "umschreiben" bzw anpassen, sodass drehungen möglich sind etc
