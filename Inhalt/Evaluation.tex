\chapter{Evaluation}
\section{Ergebnis}
%beide
Insgesamt kam es im Verlauf der Arbeit zu einer Vielzahl von unerwarteten Schwierigkeiten und Verzögerungen. Die Anfangsphase war hauptsächlich von technischen Hardwareproblemen geprägt. Dabei war zuerst die Drohne defekt und anschließend ist die Grafikkarte des Laptops ausgestiegen. \newline
Auch softwareseitig kam es zu Verzögerungen. Die externen Projekte SVO und REMODE setzen eine sehr spezielle Umgebungskonfiguration voraus. So musste durch einen iterativen Prozess die richtige Kombination aus Betriebssystemversion, ROS Distribution und Grafikkartentreiber herausgefunden werden. Hinzu kommen eine Vielzahl von nötigen Dependencies, wie beispielsweise NVIDIA CUDA.  \newline
Auch die Kalibrierung der Kamera und die Konfiguration der Parameter war eine Herausforderung. Dies war das Resultat der Abweichenden Projektbedingungen, wie die nach vorne gerichtete Kamera, oder die simulierte Drohne. \newline
Trotzdem konnten am Ende erfolgreich Tiefenbilder aus der Simulation gewonnen werden. Die Anforderung, die Gestensteuerung der Drohne von C\# unter Windows auf C++ unter Ubuntu zu migireren konnte ebenso erfolgreich umgesetzt werden. \newline
Als Folge der Verzögerungen war der Zeitraum für die Implementation eines Assistenzsystems sehr klein. Damit war es nur noch möglich das Thema theoretisch auszuarbeiten.

\section{Ausblick}
\subsection{Andere Simulatoren}
Die Simulationsumgebung Gazebo ist nicht die einzige verfügbare zur Simulation von Quadrocoptern. Jedoch bringt sie durch die einfache Anbindung von ROS einiges an Vorteilen mit sich. Damit gehen allerdings auch Nachteile einher. So sind die Kameraeingaben nicht sehr realistisch und Szenarien die in der Simulation funktionieren müssen in der Realität nicht funktionieren. Ebenso ist das Flugverhalten in manchen Situation nicht realitätsgetreu und kann zu verfälschten Ergebnisse führen. Um diesem Vorzubeugen ist es ratsam die Resultate mit anderen Simulationen vergleichen. Eine aktuelle Simulationsumgebung zur Simulation von Quadrocoptern ist der AirSim von Microsoft.\cite{airsim} Ursprünglich entwickelt um Trainingsdaten zum maschinellem Lernen sammeln, kann er ebenfalls auch für herkömmliche Simulation verwendet werden. Aktuell ist allerdings nur für Windows Betriebssysteme verfügbar. \cite{airsimpaper} Er biete eine fotorealistische Umgebung und ein akkurates Flugverhalten, dadurch ist besonders für Demos und Showcases besser geeignet. \newline
Es existieren weiterhin andere Simulatoren, allerdings sind die meisten spielerisch veranlagt und bieten keine programmatische Schnittstelle, weshalb sie für den Zweck der Studienarbeit nicht sinnvoll verwendet werden könne
%beide



\subsection{Assistenzsystem}
%genaue Untersuchung mit PCL algorithmen
%svo und remode "umschreiben" bzw anpassen, sodass drehungen möglich sind etc